\begin{frame}
    \frametitle{\problemtitle}
    \begin{itemize}
        \item<+-> Le volume total du parallélépipède rectangle est $h\times l\times p$
        \item<+-> Le volume pris par la roche est la somme de chaque matrice, on a donc du $\mathcal{O}(l\cdot p)$.
        \item<+-> Pour éviter de dépasser la hauteur, il suffit de prendre le maximum entre la hauteur et la somme.
    \end{itemize}
    % \solvestats
\end{frame}
