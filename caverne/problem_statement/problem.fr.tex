\problemname{}

\illustration{0.3}{caverne.jpg}{
    Caverne. CC BY-SA 4.0
}

Vous êtes un spéléologue et votre métier est d'étudier les Kryptes Antiques Resultant de Wombats Archaïques (KARWA). Le but de votre travail est d'estimer le volume libre dans ces cavernes mystérieuses afin de les inonder pour créer des nappes phréatiques et y stocker de l'eau ! Pour cela, vous possédez un relevé topographique du sol et du plafond. Ces données vous viennent sous cette forme :
\begin{itemize}
\item Trois entiers $h$, $l$ et $p$ représentant la hauteur, la longueur et la profondeur d'un parallélépipède rectangle représentant la zone scannée.
\item Une matrice de taille $l\times p$ représentant la topographie du plafond, chaque entrée coefficient étant la hauteur du plafond en partant du haut du parallélépipède de $1m^2$.
\item Une matrice de taille $l\times p$ représentant la topographie du sol, chaque entrée coefficient étant la hauteur du sol en partant du bas du parallélépipède de $1m^2$.
\end{itemize}

On considère que si la hauteur du plafond et celle du sol sur le même mètre carré se croisent (autrement dit, leur somme est plus grande que la hauteur du parallélépipède) alors la caverne forme un pilier et ne peut donc pas accueillir d'eau à cet endroit.

\begin{Input}
    L'input consiste en :
    \begin{itemize}
        \item Une ligne avec les trois entiers $h$, $l$ et $p$ ($1 \leq h, l, p \leq 1000$), représentant respectivement la hauteur, la longueur et la profondeur de la caverne.
        \item $l$ lignes, chacune contenant $p$ entiers entre $0$ et $h$, pour la matrice représentant la topographie du plafond.
        \item $l$ lignes, chacune contenant $p$ entiers entre $0$ et $h$, pour la matrice représentant la topographie du sol.
    \end{itemize}
\end{Input}

\begin{Output}
    Le volume de la caverne.
\end{Output}
