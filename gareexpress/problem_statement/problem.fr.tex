\problemname{}

\illustration{0.3}{rail.jpg}{}% Source: pixabay.org

En raison d'une grève ce 1er avril 2025, la SNCB a lancé son plan KARWA (Kilométrage Ajusté pour les Retards Wallons Aléatoires). C'est un mystérieux phénomène ferroviaire qui affecte tous les trains circulant entre Louvain-la-Neuve et Mons.

Elena, une étudiante distraite, s'est inscrite au fameux concours KARWa à Mons au lieu de Louvain-la-Neuve ce jour-là. Elle doit absolument prendre un train pour rejoindre Mons à temps. Malheureusement, elle vient de rater son train initial, et elle doit attendre le prochain train disponible.

Un train part toutes les $X$ minutes, et Elena est arrivée sur le quai à la minute $N$. Elle veut savoir à quelle minute partira le prochain train.

\begin{Input}
    L'entrée consiste en :
    \begin{itemize}
        \item Une ligne contenant un entier $N$ ($0 \leq N \leq 10^{9}$), la minute à laquelle Elena est arrivée sur le quai.
        \item Une ligne contenant un entier $X$ ($1 \leq X \leq 10^{9}$), la fréquence en minute d'un train partant pour Mons.
    \end{itemize}
\end{Input}

\begin{Output}
    Un entier représentant l'heure en minutes du prochain train vers Mons dans lequel Elena pourra entrer.
\end{Output}
