\problemname{lettre}

%\illustration{0.3}{image.jpg}{
%    Caption of the illustration (optional).
%    CC BY-SA 4.0 by X on \href{https://example.com/reference-to-image}{Y}
%}

% optionally define variables/limits for this problem
\newcommand{\maxa}{123456789}

% TODO: Remove this comment when you're done writing the problem statement.
Dans le monde des Kiwis Arrangés Régulièrement Wallons et Agressifs, les kiwis sont très soucieux de leur consommation d'énergie et cherchent à optimiser tout ce qui est optimisable. Les postiers de ce monde ont donc horreur de devoir repasser deux fois par la même rue lors de leur tournée. Votre travail est donc de les rassurer en leur affirmant si oui ou non, il est possible d'avoir un chemin parfait. On ne vous demandera de ne regarder que les rues accessibles depuis votre point de départ. De plus, les kiwis n'aiment pas les sens uniques et donc toutes les rues sont empruntables dans les deux sens. \\
Pour ce faire, la ville vous fournit son plan de la façon suivante :
\begin{itemize}
\item les carrefours numméroté de 0 à $c-1$ où $c$ est le nombre de carrefours dans la ville
\item les rues reliant deux carrefours (attention, plusieurs routes peuvent relier deux même carrefours)
\end{itemize}
Les kiwis étant des animaux peu conscient de leur propre bêtise, il est possible qu'un carrefour soit relié à $0$, $1$ ou $2$ rues, ce qui n'est pas vraiment considéré comme un carrefour dans notre monde.
Le trajet entre le point de départ de la tournée et le bureau de poste n'est pas comptabilisé. 
\begin{Input}
    The input consists of:
    \begin{itemize}
        \item Une ligne contenant l'entier $1 \leq c \leq n$ correspondant au nombre total de carrefours dans la ville où vous travaillez.
        \item Un entier $0\leq d \leq c$ le carrefour où se trouve votre bureau de poste.
        \item Une ligne contenant l'entier $0 \leq r \leq n$ correspondant au nombre de rues.
        \item $r$ lignes contenant deux entiers $0\leq a,b < c$ reliant les carrefours $a$ et $b$.
    \end{itemize}
\end{Input}

\begin{Output}
    $Oui$ s'il existe un chemin qui passe une et une seule fois par toutes les rues (peu importe la position du bureau de poste, il peut existe un tel chemin sans que le point de départ soit le bureau de poste). $Non$ dans le cas contraire
\end{Output}
