\begin{frame}
    \frametitle{\problemtitle}
    \begin{block}{Problème}
        Déterminer si la composante connexe du bureau de poste admet un parcours eulérien (passant une seule fois sur chaque arête).
    \end{block}
    \begin{itemize}
        \item<+-> Solution naïve, essayer tous les chemins jusqu'à trouver le bon. C'est trop naïf, il faut faire plus fin.
        \item<+-> Un graphe admet un parcours eulérien si et seulement s'il possède soit $2$, soit $0$ sommets avec un nombre d'arêtes impair.
        \item<+-> Une solution en $\mathcal{O}(n)$ est donc de précalculer le nombre d'arêtes pour chaque sommet, puis d'effectuer un BFS/DFS et de compter le nombre de sommets avec un nombre impair d'arêtes.
    \end{itemize}
    % \solvestats
\end{frame}
