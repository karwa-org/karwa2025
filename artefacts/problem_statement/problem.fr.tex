\problemname{\problemyamlname{}}

%\illustration{0.3}{}{}

todo

Le rituel fonctionne comme suit :

    Il y a $n$ artefacts, chacun avec une valeur énergétique $A_i$.

    Vous pouvez fusionner deux artefacts adjacents pour en créer un nouveau, dont la valeur énergétique est la somme des deux artefacts d'origine.

    Chaque fusion coûte exactement cette somme d'énergie.

    L'objectif est de fusionner tous les artefacts en un seul en minimisant le coût total du rituel.

Exemple

Si on a $4, 3, 2, 6$, voici un exemple de séquence de fusions optimales :

    Fusionner $3$ et $2$ $\rightarrow$ nouveau artéfacts : $4, 5, 6$, coût 5

    Fusionner $4$ et $5$ $\rightarrow$ nouveau artéfacts : $9, 6$, coût 9

    Fusionner $9$ et $6$ $\rightarrow$ un unique artéfact $15$, coût 15

Coût total minimal : $5 + 9 + 15 = 29$.

\begin{Input}
    L'entrée consiste en :
    \begin{itemize}
        \item Une ligne contenant un entier $n$ ($1 \leq n \leq 3000$), le nombres d'artéfacts.
        \item $n$ lignes, chacune contenant un entier entre $1$ et $10^{9}$ qui représente l'énergie de l'artéfact..
    \end{itemize}
\end{Input}

\begin{Output}
    La plus petite quantité d'énergie nécessaire pour combiner tous les artéfacts.
\end{Output}
