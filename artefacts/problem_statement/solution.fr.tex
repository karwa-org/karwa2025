\begin{frame}
    \frametitle{\problemtitle}
    \begin{block}{Problème}
        Étant donné une liste de $n$ éléments, on peut fusionner deux éléments adjacents pour un coût égal à la somme des deux éléments. On aimerait fusionner la liste en un seul élément en minimisant le coût nécessaire.
    \end{block}
    \pause
    \begin{itemize}[<+->]
        \item Approche naïve 1 : tester toutes les découpes possibles. Complexité exponentielle !
        \item Approche naïve 2 : on regarde toutes les paires d'éléments $(a_i,a_{i+1})$ et on fusionne la paire dont la somme est minimale, puis on recommence. Ok si deux paires ne sont jamais équivalences, mais en cas d'égalité, un choix doit être fait et tout explorer fait tomber dans l'approche naïve 1.
        \item Solution : Dynamic Programming ! Pour fusionner les éléments de $i$ à $j$, on trouve le meilleur $k$ qui sépare $[i,j]$ en $[i,k]$ et $[k+1,j]$. On doit ensuite ajouter la somme des éléments entre $i$ et $j$. \\
        $\rightarrow$ DP[i][j] représente le meilleur coût pour fusionner la liste de $i$ à $j$ :
        \[
        DP[i][j] = \min_{k}(DP[i][k] + DP[k+1][j] + \texttt{sum}[i][j]).
        \]
        On peut pré-calculer la somme des éléments entre $i$ et $j$ via la somme des préfixes, puis $\texttt{sum}[i][j]) = \texttt{prefix}[j+1] - \texttt{prefix}[i]$.
        \item Complexité : $\mathcal{O}(n^3)$.
    \end{itemize}

    %\pause\solvestats
\end{frame}