\problemname{\problemyamlname{}}

\illustration{0.3}{vecteezy_exploring-industrial-pipe-systems-underwater-ocean-depths_57834830.jpeg}{
    From \href{https://www.vecteezy.com/free-photos/pipes}{vecteezy}
}

Le KARWA (Kanalisation Aquatic Route for Wayward Adventurers) est un réseau complexe de canalisations souterraines conçu pour transporter de l'eau vers l'océan, sous la forme d'une grille $n \times m$. Cependant, il arrive parfois qu'un poisson s'y retrouve coincé !

Aujourd'hui, c'est le jeune Nemo qui s'est égaré dans ce dédale de tuyaux et il a besoin de votre aide pour rejoindre l'océan, mais certaines intersections de tuyaux sont soumises à des courants perturbateurs qui modifient son effort de nage.

Nemo peut se déplacer horizontalement ou verticalement dans le réseau, cela lui prend initialement $1$ minute de nage par unité de distance. Mais lorsqu'il passe dans une intersection perturbatrice, l'effort nécessaire pour en sortir ajoute $3$ minutes au trajet (donc $4$ minutes au total pour s'échapper d'une perturbation).

Nemo connaît sa position initiale $(i_s, j_s)$ et la position de la sortie $(i_t, j_t)$. Il veut y arriver le plus vite possible.

\begin{Input}
    L'entrée consiste en :
    \begin{itemize}
        \item Une ligne contenant deux entiers $n, m$ $(1 \leq n, m \leq 700)$ représentant la taille de la grille de $n$ lignes et $m$ colonnes.
        \item Une ligne contenant quatre entiers $i_s, j_s, i_t, j_t$ $(0 \leq i_s, i_t < n, 0 \leq j_s, j_t < m)$, les positions de départ $(i_s,j_s)$ et d'arrivée $(i_t,j_t)$,
        \item $n$ lignes, chacune contenant $m$ caractères qui représentent le type d'intersection : \texttt{+} pour une intersection sans perturbation, \texttt{P} pour une intersection avec perturbation.
    \end{itemize}
\end{Input}

\begin{Output}
    Un entier correspondant au temps minimal pour atteindre la sortie.
\end{Output}