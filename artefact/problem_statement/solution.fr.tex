\begin{frame}
    \frametitle{\problemtitle}
    \begin{block}{Problème}
        Étant donné une liste de $n$ nombres, fusionner la liste en un seul élément en minimisant le coût nécessaire (en fusionnant paire par paire, où le coût d'une fusione est la somme des nombres).
    \end{block}
    \pause
    \begin{itemize}
        \item Approche naïve 1 : tester toutes les découpes possibles. Complexité exponentielle ! \pause
        \item Approche naïve 2 : chercher la paire $(a_i,a_{i+1})$ qui minimise $a_i+a_{i+1}$. Comment départager les égalités ? On retombe dans l'approche naïve 1.
    \end{itemize}
    \pause
    \begin{block}{Solution : Dynamic Programming (bottom-up)}
        Pour fusionner les éléments de $i$ à $j$, on trouve le meilleur $k$ qui sépare $[i,j]$ en $[i,k]$ et $[k+1,j]$. On doit ensuite ajouter la somme des éléments entre $i$ et $j$. \\ \pause
        $\rightarrow$ DP[i][j] représente le meilleur coût pour fusionner la liste de $i$ à $j$ :
        \[
        DP[i][j] = \min_{i<k<j}(DP[i][k] + DP[k+1][j] + \texttt{sum}[i][j]).
        \]
        \pause
        On peut pré-calculer la somme des éléments entre $i$ et $j$ via la somme des préfixes : $\texttt{sum}[i][j] = \texttt{prefix}[j+1] - \texttt{prefix}[i]$.
    \end{block}
\end{frame}

\begin{frame}
    \frametitle{\problemtitle}
    \begin{block}{Solution similaire : Memo\"isation (top-down)}
       On définit une fonction récursive \texttt{solve(i,j)} jouant le rôle $DP[i][j]$ et on retient via mémoïsation les paires $(i,j)$ déjà traitées. Il suffit ensuite d'appeler $solve(0,n-1)$.
    \end{block}
    \pause
    Complexité ces solutions : $\mathcal{O}(n^3)$.

    \solvestats
\end{frame}